%%%%%%%%%%%%%%%%%%%%%%%%%%%%%%%%%%%%% BEGIN HEADERS %%%%%%%%%%%%%%%%%%%%%%%%%%%%%%%%%%%%%%%%%%%%%%%%%%%%%
\documentclass[11pt,conference]{IEEEtran}

\usepackage{longtable}
\usepackage{graphicx}
\usepackage[utf8]{inputenc}
\usepackage{fancyhdr}
\usepackage{float}
\usepackage[hidelinks]{hyperref}
\usepackage{listings}
\usepackage{color}
\usepackage{natbib}

% Your names in the header
\pagestyle{fancy}
\rhead{Enrico Tedeschi}
\lhead{INF-3200 Distributed Systems - Assignment 1}
\cfoot{\thepage}

% Used for including code in a stylized manner
\definecolor{codegreen}{rgb}{0,0.6,0}
\definecolor{codegray}{rgb}{0.5,0.5,0.5}
\definecolor{codepurple}{rgb}{0.58,0,0.82}
\definecolor{backcolour}{rgb}{0.95,0.95,0.92}
 

\lstdefinestyle{mystyle}{
    backgroundcolor=\color{backcolour},   
    commentstyle=\color{codegreen},
    keywordstyle=\color{magenta},
    numberstyle=\tiny\color{codegray},
    stringstyle=\color{codepurple},
    basicstyle=\footnotesize,
    breakatwhitespace=false,         
    breaklines=true,                 
    captionpos=b,                    
    keepspaces=true,                 
    numbers=left,                    
    numbersep=5pt,                  
    showspaces=false,                
    showstringspaces=false,
    showtabs=false,                  
    tabsize=2
}

\lstset{style=mystyle}

% The Title
\title{UiT INF-3200 Distributed Systems - Project 1\\Fall 2015}

% Your name and email
\author{Enrico Tedeschi\\ete011@post.uit.no
    \and Mike Murphy\\mmu019@post.uit.no}


%%%%%%%%%%%%%%%%%%%%%%%%%%%%%%%%%%%%% END HEADERS %%%%%%%%%%%%%%%%%%%%%%%%%%%%%%%%%%%%%%%%%%%%%%%%%%%%%

\begin{document}

% Create the title and everything
\maketitle


\section{Introduction}

Our task was to implement a simple distributed key-value store.


\subsection{Requirements}

% TODO: Multiple nodes.
% TODO: Frontend node connects randomly.

\if 0

\begin{itemize} 
\item Design and implement a distributed key-value storage system
\item Run the system on the uvocks cluster (urocks.cs.uit.no)
\end{itemize}

\fi


\section{Technical Background}

\if 0

\begin{itemize} 
\item[--] Distributed systems concept
\item[--] Basic programming approach
\item[--] Knowledge of Python language
\item[--] Notion of design pattern principles
\item[--] Theory about software engineering
\item[--] Knowledge of git to manage the software versions
\item[--] Notion of Chord architecture
\item[--] Knowledge of programming with API in Python
\item[--] Basic approach to Linux command line principles
\end{itemize}

\fi

\if 0
The interesting thing about DHTs is that storage and lookups are distributed among multiple machines \cite{linuxjournal_dht}
Chord is an efficient distributed lookup system based on consistent hashing. It provides a
unique mapping between an identifier space and a set of nodes \cite{chord}
\newline
Chord is efficient: determining the successor of an identifier requires that $O(log N)$ messages be exchanged with high probability where $N$ is the number of servers in the
Chord network. Adding or removing a server from the network
can be accomplished, with high probability, at a cost of $O(log^2 N)$ messages.
\cite{chord}
\fi


\section{Design}

% TODO: Chord's one-dimensional key-space, without finger tables.
% TODO: Nodes fixed, so key space divided evenly.
% TODO: How to block. Synchronous at first.


\section{Implementation}

% TODO: Existing frontend in Python
% TODO: Shell script to start
% TODO: HTTP API

\if
Implemented on the file \textit{startup.sh} the possibility to choose with a given argument how many nodes to run. Since the Linux's bash works as a pipeline, each command has been given using the input of the output of the previous one, and at the end it gets the random nodes on the cluster according to a number given in input while executing the script.
\begin{lstlisting}
nodes = 
\$(rocks list host |
grep compute |
cut -d" " -f1 |
sed 's/.\$//' |
shuf |
head -n "\$num_hosts")
\end{lstlisting}
\fi



\section{Discussion}

% TODO: O(n) lookup.
% TODO: Finger tables O(log n)



\section{Evaluation}

% TODO: time actual implementation


\section{Conclusion}

% TODO: Conclusion


\section{CITATION PLACEHOLDER}

% TODO: Remove section
This text is only here to hang some citations before they're used in the real
text.
\cite{linuxjournal_dht}
\cite{chord}
\cite{parallel}


\bibliographystyle{plain}
\bibliography{report}


\end{document}
